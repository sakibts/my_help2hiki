\subsubsection{hikiとは}
Rubyで書かれた高機能・高速Wikiクローン.
参考文献3(\verb|http://hikiwiki.org/ja/| 1/27)
CGI(Commond Gateway Interface)を利用して,Webサーバと連動して動く.
参考文献4(\verb|https://ja.wikipedia.org/wiki/Hiki| 1/27)
西谷研究室では,hikiの形式を利用してサイトを作り,
研究室内での情報共有やgemの使い方などを掲載して閲覧できるようにしている.
また,卒業論文の作成にもhikiの形式で作成している.

