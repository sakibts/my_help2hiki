\subsubsection{hiki形式で書かれた文章をlatex形式に自動変換する}
\paragraph{コマンド}\begin{quote}\begin{verbatim}
Usage: hiki2latex [options] FILE
    -v, --version                    show program Version.
    -l, --level VALUE                set Level for output section.
    -p, --plain FILE                 make Plain document.
    -b, --bare FILE                  make Bare document.
        --head FILE                  put headers of maketitle file.
        --pre FILE                   put preamble file.
        --post FILE                  put post file.
        --listings                   use listings.sty for preformat with style.
\end{verbatim}\end{quote}
上記のようなコマンドがある.

\paragraph{使用例}
\begin{itemize}
\item hiki2latex -v
\end{itemize}
hiki2latexのversionを表示する.
実行例
\begin{quote}\begin{verbatim}
/Users/saki% hiki2latex -v
hiki2latex 0.9.12
\end{verbatim}\end{quote}
\begin{itemize}
\item hiki2latex -p
\end{itemize}
hiki形式のsample,hiki2latex\_sample.hikiを例とする.
\begin{quote}\begin{verbatim}
/Users/saki/my_help2hiki/my_help2hiki_saki% cat hiki2latex_sample.hiki 
!title1
!!subtitle1.1
*list1
*list2
!!subtitle1.2
*list1

!title2
\end{verbatim}\end{quote}
コマンドを実行すると以下のようになる.
\begin{quote}\begin{verbatim}
/Users/saki/my_help2hiki/my_help2hiki_saki% hiki2latex -p hiki2latex_sample.hiki
\documentclass[12pt,a4paper]{jsarticle}
\usepackage[dvipdfmx]{graphicx}
\begin{document}
\section{title1}
\subsection{subtitle1.1}\begin{itemize}
\item list1
\item list2
\end{itemize}
\subsection{subtitle1.2}\begin{itemize}
\item list1
\end{itemize}
\section{title2}
\end{document}
\end{verbatim}\end{quote}
このコマンドにより,hiki形式の雛形が生成されていることが分かる.

\begin{itemize}
\item hiki2latex -b
\end{itemize}
hiki2latex\_sample.hikiを例とするコマンド実行例
\begin{quote}\begin{verbatim}
/Users/saki/my_help2hiki/my_help2hiki_saki% hiki2latex -b hiki2latex_sample.hiki
\section{title1}
\subsection{subtitle1.1}\begin{itemize}
\item list1
\item list2
\end{itemize}
\subsection{subtitle1.2}\begin{itemize}
\item list1
\end{itemize}
\section{title2}
\end{verbatim}\end{quote}
上記の間に記述された部分のみを生成する.
同じファイルの中に,複数のlatexのファイルを含んでいるときに
個別でファイル作成をするときに便利.

\begin{itemize}
\item hiki2latex --head : タイトル,著者を記述するときのフォーマットを挿入する.
\item hiki2latex --pre : 文字の前の空白などを入れる.
\item hiki2latex --post
\end{itemize}
上記3つは全てフォーマットを標準と違うものにするためのコマンド.

