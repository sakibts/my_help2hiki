\subsubsection{TeXとは}
「テック」または「テフ」と読む,組版ソフト.
スタンフォード大学のDonald E.Knuth教授によって作られた.
他のソフトと組み合わせて,組版結果を画面や紙に出力したり,
PDF形式で出力したりする.

\subsubsection{TeXの特徴}
\begin{itemize}
\item フリーソフトなので無料で入手でき,自由に中身を調べたり改良したりできる.
\item 商品利用が自由.
\item Windows,Mac,LinuxなどのOSに関わらず同じ動作をする.
\item TeXの文書はテキストファイルなので,普通のテキストエディタで読み書きでき,再利用やデータベース化が容易.
\item 数式の組版に定評があり,数式をテキスト形式で表す標準となっている.
\end{itemize}
\subsubsection{latexとは}
「ラテック」または「ラテフ」と読む.
DEC(現在のHP)のコンピュータ科学者Leslie Lamportによって機能強化されたTeX.
もとのTeXと同様にフリーソフトとして配布されている.

参考文献5「LATEX2e 美文書作成入門」,奥村晴彦/黒木祐介著,技術評論社,2016,p.1-3

