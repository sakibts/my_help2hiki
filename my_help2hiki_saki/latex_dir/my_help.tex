\subsubsection{ユーザ独自のhelpを作成するgem}
\paragraph{gemとは}
正式名称はRubyGems.
Ruby用のライブラリを使う時に必要となるソフトウェアのこと.
パッケージ管理ツールgemがあることで,Ruby用ライブラリの
インストール,アンインストール,バージョン管理などを簡単に
行うことができる.
プログラミング言語Rubyのファイルに付属されていて,
無料で利用することができる.

\paragraph{gemの利点}
\begin{itemize}
\item 標準化された構造があるので,初めてみた人でも分かるようになっている.
\item gemがあることで,簡単にRuby用ライブラリをインストールでき,初心者でもアプリ機能を装備できる.
\item 誰でも作成,配布が可能である.
\end{itemize}
\paragraph{my\_helpとは}
西谷研究室で使用しているgem.
\begin{quote}\begin{verbatim}
Usage: my_help [options]
    -v, --version                    show program Version.
    -l, --list                       list specific helps
    -e, --edit NAME                  edit NAME help(eg test_help)
    -i, --init NAME                  initialize NAME help(eg test_help).
    -m, --make                       make executables for all helps.
    -c, --clean                      clean up exe dir.
        --install_local              install local after edit helps
        --delete NAME                delete NAME help
        --hiki                       my_help2hiki
\end{verbatim}\end{quote}
上記のようなコマンドが用意されていて,
ユーザ独自のhelp(メモ)を作成することができる.

\paragraph{my\_helpを研究室内で利用する利点}
\begin{itemize}
\item 研究室内でのメモの書き方が統一できる.
\item どこにメモをしたか忘れることがない.
\item 普段研究の為に使うターミナルから離れること無くメモを残すことができるので,書きたいときにすぐに書くことができる.考文献(\verb|https://blog.codecamp.jp/rails-gem| 1/27)
\end{itemize}
