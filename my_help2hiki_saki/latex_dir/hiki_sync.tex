\paragraph{gem中のhikisからhikiへの自動変換}
rubyのライブラリーパッケージの標準となるgemのdirectory構造にhikisというdirectoryを作って文書作成している.hiki --initializeでこのなかのhikidocファイルをウェブ上のhikiと同期する機能を提供する.
これによって,gem/hikisで作成した文書は,githubあるいはrubygems.orgを通じて共有可能となる.
以下にこの同期をスムーズに行うための幾つかのconventionを使用法とともにまとめている.

\paragraph{使用法,コマンド}
\begin{itemize}
\item hiki --initializeで必要なファイル(Rakefile, ./.hikirc, hiki\_help.yml)がcopyされる
\item hiki\_help.ymlは適宜~/.my\_helpにcopyしてhiki\_helpとして利用\verb|my_help参照(MyHelp_install)|
\item rake syncによってhikiディレクトリーと同期が取られる
\item rake convert 20 TARGET.pngによって,figs/TAERGET.pngに20%縮小して保存される
\item hiki -u TARGETによってブラウザー表示される
\end{itemize}
\paragraph{同期に関する制約}
\begin{itemize}
\item hikiはフラットなdirectory構造を取っている
\item hikiの文書はスネーク表記(例えば,latex2hiki\_saki)で階層構造に似せている
\item hikiのurlの接頭語となる名前をbasenameのdirectory名とする.
\item directory名が'hikis'である場合はその親directory名となる.
\item ~/.hikircのtarget directoryを同期先のdirectoryとする.
\item ~/.hikircがない場合は同期先のdirectoryを聞く.
\item それらは./.hikircに保存される
\end{itemize}
\paragraph{テキスト}
\begin{itemize}
\item テキストの拡張子は'.hiki'としている
\item hikiでのurlはテキスト前とディレクトリーから自動生成される
\item 例えば,hiki2latex\_saki/introduciton.hikiとするとhiki2latex\_saki\_introducitonと変換される
\item attach\_anchorでは
\end{itemize}\begin{quote}\begin{verbatim}
'{{attach_anchor(test.png, hiki2latex_saki)}}'
\end{verbatim}\end{quote}
と,directory指定しなければならない.

