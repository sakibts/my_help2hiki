先ほど記述した欠点をふまえて,
my\_help2hikiに既存システムであるhiki2latexを組み込み
\textbf{my\_help2hikiによるmemo,hiki,latexの関係}の図の破線部分にあたるシステムを作ることで,
my\_helpからlatexへとすぐに変換することのできるような,
my\_help,hiki,latexをさらに強く結びつけるシステムができるのではないかと考えている.
また,西谷研究室には内部サイトがあり,研究室内で使うシステムのマニュアルなどが公開されている.
hiki形式への変換ができればwikiで表示することはできるようになるが,
my\_helpをhikiやlatexに変換するのと同じように,その内部サイトに
各研究室生がmy\_helpによって作成したmemoを表示できるコマンドを追加すれば,
さらに便利なシステムになると考えられる.

