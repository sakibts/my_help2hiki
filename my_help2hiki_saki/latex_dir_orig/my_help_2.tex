\section{関連する先行研究}
\subsection{my\_help}
西谷研究室で使用しているユーザ独自のメモを作成するgem.

\paragraph{gemとは}
\begin{description}
\item 正式名称はRubyGems.
Ruby用のライブラリを使う時に必要となるソフトウェアのこと\cite{c}.
パッケージ管理ツールgemがあることで,Ruby用ライブラリの
インストール,アンインストール,バージョン管理などを簡単に
行うことができる.
プログラミング言語Rubyのファイルに付属されていて,
無料で利用することができる.
\end{description}

\paragraph{gemの利点}
\begin{itemize}
\item 標準化された構造があるので,初めてみた人でも分かるようになっている.
\item gemがあることで,簡単にRuby用ライブラリをインストールでき,初心者でもアプリ機能を装備できる.
\item 誰でも作成,配布が可能である.
\end{itemize}

\paragraph{コマンド}
\begin{description}
\item 
\end{description}
\begin{quote}\begin{verbatim}
Usage: my_help [options]
    -v, --version                    show program Version.
    -l, --list                       list specific helps
    -e, --edit NAME                  edit NAME help(eg test_help)
    -i, --init NAME                  initialize NAME help(eg test_help).
    -m, --make                       make executables for all helps.
    -c, --clean                      clean up exe dir.
        --install_local              install local after edit helps
        --delete NAME                delete NAME help
        --hiki                       my_help2hiki
\end{verbatim}\end{quote}

\paragraph{my\_helpを研究室内で利用する利点}
\begin{itemize}
\item 研究室内でのメモの書き方が統一できる.
\item どこにメモをしたか忘れることがない.
\item 普段研究の為に使うターミナルから離れること無くメモを残すことができるので,書きたいときにすぐに書くことができる.
\end{itemize}

\subsection{hiki}
Rubyで書かれた高機能・高速Wikiクローン\cite{d}.
CGI(Commond Gateway Interface)を利用して,Webサーバと連動して動く\cite{e}.
西谷研究室では,hikiの形式を利用してサイトを作り,
研究室内での情報共有やgemの使い方などを掲載して閲覧できるようにしている.
また,卒業論文の作成にもhikiの形式で作成している.

\subsection{hikiutils}
gem中のhikisからhikiへの自動変換をする.
rubyのライブラリーパッケージの標準となるgemのdirectory構造にhikisというdirectoryを作って文書作成している.hiki --initializeでこのなかのhikidocファイルをウェブ上のhikiと同期する機能を提供する.
これによって,gem/hikisで作成した文書は,githubあるいはrubygems.orgを通じて共有可能となる.
以下にこの同期をスムーズに行うための幾つかのconventionを使用法とともにまとめている.

\paragraph{使用法,コマンド}
\begin{itemize}
\item hiki --initialize : 必要なファイル(Rakefile, ./.hikirc, hiki\_help.yml)がcopyされる
\item hiki\_help.yml : 適宜~/.my\_helpにcopyしてhiki\_helpとして利用\verb|my_help参照(MyHelp_install)|
\item rake sync : hikiディレクトリーと同期が取られる
\item rake convert 20 TARGET.png : figs/TAERGET.pngに20\%縮小して保存される
\item hiki -u TARGET : ブラウザー表示される
\end{itemize}

\paragraph{同期に関する制約}
\begin{itemize}
\item hikiはフラットなdirectory構造を取っている
\item hikiの文書はスネーク表記(例えば,latex2hiki\_saki)で階層構造に似せている
\item hikiのurlの接頭語となる名前をbasenameのdirectory名とする.
\item directory名が'hikis'である場合はその親directory名となる.
\item ~/.hikircのtarget directoryを同期先のdirectoryとする.
\item ~/.hikircがない場合は同期先のdirectoryを聞く.
\item それらは./.hikircに保存される
\end{itemize}
\paragraph{テキスト}
\begin{itemize}
\item テキストの拡張子は'.hiki'としている
\item hikiでのurlはテキスト前とディレクトリーから自動生成される
\item 例えば,hiki2latex\_saki/introduciton.hikiとするとhiki2latex\_saki\_introducitonと変換される
\item attach\_anchorでは
\end{itemize}\begin{quote}\begin{verbatim}
'{{attach_anchor(test.png, hiki2latex_saki)}}'
\end{verbatim}\end{quote}
と,directory指定しなければならない.