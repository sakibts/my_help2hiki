\section{関連するソフトの振る舞い}
本章では,対象とするソフトはあまり一般に普及しているものではないため,
最初にソフト(my\_helpおよびhiki)の特徴と簡単な振る舞いを紹介する.

\subsection{my\_help}
my\_helpは西谷研究室で使用しているユーザ独自のメモを作成するgemである.

gemは正式名称をRubyGemsといい,
Ruby用のライブラリを使う時に必要となるソフトウェアのことである\cite{c}.
パッケージ管理ツールgemがあることで,Ruby用ライブラリの
インストール,アンインストール,バージョン管理などを簡単に
行うことができる.
プログラミング言語Rubyのファイルに付属されていて,
無料で利用することができる.

gemの利点は次の通りである。
\begin{itemize}
\item 標準化された構造があるので,初めてみた人でも分かるようになっている.
\item gemがあることで,簡単にRuby用ライブラリをインストールでき,初心者でもアプリ機能を装備できる.
\item 誰でも作成,配布が可能である.
\end{itemize}

my\_helpが提供するコマンドとその振る舞いをhelp表示から示す.
\begin{quote}\begin{verbatim}
Usage: my_help [options]
    -v, --version                    show program Version.
    -l, --list                       list specific helps
    -e, --edit NAME                  edit NAME help(eg test_help)
    -i, --init NAME                  initialize NAME help(eg test_help).
    -m, --make                       make executables for all helps.
    -c, --clean                      clean up exe dir.
        --install_local              install local after edit helps
        --delete NAME                delete NAME help
        --hiki                       my_help2hiki
\end{verbatim}\end{quote}

my\_helpではhelpの要素をこのhelp表示と同じ構造,項目と対応する記述で表示される.
それぞれの項目はまとまりをつくり,それを-lでリスト表示される.
--versionおよび--listはgem標準に準拠するために用意されている.
上記コマンドのhelp表示のNAMEには作成したhelpの名前が入る.
--edit NAMEにより,NAMEの内容を記述し更新する,--init NAMEにより,NAMEを初期化する.
exeのディレクトリを空にするには--cleanのコマンドで実行ができる.
helpを書いた後にmy\_helpによって記述や閲覧ができるようにするために,
--install\_localが用意されている.
NAMEのhelpを消したいときには,--delete NAMEによって消去ができる.
--hikiコマンドにより,my\_helpからhikiへの自動変換を行い,wikiでブラウザ表示を行う.
my\_helpを研究室内で利用する利点は次の通りである.
\begin{itemize}
\item 研究室内でのメモの書き方が統一できる.
\item どこにメモをしたか忘れることがない.
\item 普段研究の為に使うターミナルから離れること無くメモを残すことができるので,書きたいときにすぐに書くことができる.
\end{itemize}

\subsection{hiki}
hikiとは,Rubyで書かれた高機能・高速Wikiクローンである\cite{d}.
CGI(Commond Gateway Interface)を利用して,Webサーバと連動して動く\cite{e}.
西谷研究室では,hikiの形式を利用してサイトを作り,
研究室内での情報共有やgemの使い方などを掲載して閲覧できるようにしている.
また,卒業論文の作成もhikiの形式で作成している.
hikiの特徴として次のことが挙げられる.
\begin{itemize}
\item オリジナルWikiに似たシンプルな書式を持つ.
\item プラグインによる機能拡張ができる\cite{f}.
\item 携帯からアクセスが可能.
\item アクセス制限ができる.
\item 柔軟性が高く,手軽に始められて操作が簡単\cite{g}.
\item 閲覧者でも修正,ページの追加などの編集が行える.
\item 作成したページを自動で整理する\cite{h}.
\end{itemize}
これらの特徴から,複数の人が同じ情報をリアルタイムで共有し,編集しやすいという利点を持つ.
まとめサイトを作るのに適しているとして,多くのサイトに使われている.
代表的なものに百科事典であるWikipediaがある.
本研究で作成を行う西谷研究室の内部サイトも,各研究室内メンバーのメモ(暗黙知を形式知化したもの)のまとめサイトであり,
hikiを本研究に利用することは最適であるといえる.
