\section{結果}
\begin{quote}\begin{verbatim}
/Users/saki% my_help --list
Specific help file:
  emacs_help	:emacsのキーバインド
  memo_help	:ヘルプのサンプル雛形
  my_todo	:my_todo
  ssh_help	:sshのhelp
\end{verbatim}\end{quote}

これは私のmy\_helpの中身を書き出している.
下がemacs\_helpの中身である.

\begin{quote}\begin{verbatim}
/Users/saki% emacs_help --all
emacsのキーバインド

特殊キー操作
  c-f, controlキーを押しながら    'f'
  M-f, escキーを押した後一度離して'f'
    操作の中断c-g, 操作の取り消し(Undo) c-x u
     cc by Shigeto R. Nishitani, 2016
+emacsのキーバインド:
+
特殊キー操作
+  c-f, controlキーを押しながら    'f'
+  M-f, escキーを押した後一度離して'f'
+    操作の中断c-g, 操作の取り消し(Undo) c-x u
+     cc by Shigeto R. Nishitani, 2016:
---
-カーソル移動cursor:
+c-f, move Forwrard,    前or右へ
+c-b, move Backwrard,   後or左へ
+c-a, go Ahead of line, 行頭へ
+c-e, go End of line,   行末へ
+c-n, move Next line,   次行へ
+c-p, move Previous line, 前行へ
---
---
-ページ移動page:
+c-v, move Vertical,          次のページへ
+M-v, move reversive Vertical,前のページへ
+c-l, centerise Line,       現在行を中心に
+M-<, move Top of file,    ファイルの先頭へ
+M->, move Bottom of file, ファイルの最後尾へ
---
---
-ファイル操作file:
+c-x c-f, Find file, ファイルを開く
+c-x c-s, Save file, ファイルを保存
+c-x c-w, Write file NAME, ファイルを別名で書き込む
---
---
-編集操作edit:
+c-d, Delete char, 一字削除
+c-k, Kill line,   一行抹消,カット
+c-y, Yank,        ペースト
+c-w, Kill region, 領域抹消,カット
+領域選択は,先頭or最後尾でc-spaceした後,最後尾or先頭へカーソル移動
+c-s, forward incremental Search WORD, 前へWORDを検索
+c-r, Reverse incremental search WORD, 後へWORDを検索
+M-x query-replace WORD1 <ret> WORD2:対話的置換(y or nで可否選択)
---
---
-ウィンドウ操作window:
+c-x 2, 2 windows, 二つに分割
+c-x 1, 1 windows, 一つに戻す
+c-x 3, 3rd window sep,縦線分割
+c-x o, Other windows, 次の画面へ移動
---
---
-バッファー操作buffer(すでにopenしてemacsにバッファーされたfile):
+c-x b, show Buffer,   バッファのリスト
+c-x c-b, next Buffer, 次のバッファへ移動
---
---
-終了操作quit:
+c-x c-c, Quit emacs, ファイルを保存して終了
+c-z, suspend emacs,  一時停止,fgで復活
\end{verbatim}\end{quote}

このように各学生のメモを何種類も作成できるが,自分のパソコンでしか見ることができなかった.
本研究によるmy\_help2hikiを使うことで,図6のようにmy\_helpをwikiで表示可能になり,
各学生のメモを研究室内で共有することができるようになる.
さらに4章で述べたような設計ができれば,研究室内のナレッジマネジメントは
今より推進させると考えている.
